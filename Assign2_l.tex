\documentclass[journal,12pt,twocolumn]{IEEEtran}
\usepackage{amsmath}
\newcommand{\myvec}[1]{\ensuremath{\begin{pmatrix}#1\end{pmatrix}}}
\title{Assignment 2}
\author{PARIMISETTY HARINADHA (CS19RESCH11004)}
\begin{document}
\maketitle
\newpage
\begin{abstract}
This document calculate the $n^{th}$ power of given matrix A, using Cayley Hamilton theorem.
\end{abstract}
Download all python codes from 
\framebox[1\width]{ https://github.com/cs19resch11004/hari }
Download all Latex-tikz codes from 
\framebox[1.1\width]{ https://github.com/cs19resch11004/hari} 
\section{Problem}
If $A$ = \myvec{ 3 & -4 \\ 1 & -1 } then prove that $A^n$ = \myvec{ $1 + 2n$ & $ -4n$ \\ $n$ & $1 - 2n$ }, where n is any positive integer.
\section{Solution}
Characteristic equation of A is $\lambda^2 - 2\lambda + 1 = 0$.
By using Cayley Hamolton theorem, given matrix A is a solution of its own characteristic equation. So $A^2 - 2A + I = 0$.
\begin {align}
	A^2 &= 2A - I  \\
	A^3 &= 3A - 2I \\
        A^4 &= 4A - 3I \\
	A^5 &= 5A - 4I
\end{align}
 By the substitution, 
\begin {align}
	A^n &= nA - (n-1)I 
\end{align}
Substituting matrix A = \myvec{ 3 & -4 \\ 1 & -1 } in above equation,
\begin {align}
	A^n &= n\myvec{ 3 & -4 \\ 1 & -1 } - (n-1)\myvec{ 1 & 0 \\ 0 & 1 } \\
	    &=\myvec{ 3n & -4n \\ n & -n } - \myvec{ (n-1) & 0 \\ 0 & (n-1)} \\
	    &=\myvec{ 2n+1 & -4n \\ n & 1-2n }  
\end{align}
So, $A^n =\myvec{ 2n+1 & -4n \\ n & 1-2n }$  
\end{document}
