\documentclass[journal,12pt,twocolumn]{IEEEtran}
\usepackage{amsmath}
\newcommand{\myvec}[1]{\ensuremath{\begin{pmatrix}#1\end{pmatrix}}}
\title{Assignment 1}
\author{PARIMISETTY HARINADHA (CS19RESCH11004)}
\begin{document}
\maketitle
\newpage
\begin{abstract}
This document explains the concept of Normal vector, Direction Vector and Y-intercept of a straight line by solving number of problems.
\end{abstract}
Download all python codes from 
\framebox[1\width]{ https://github.com/cs19resch11004/hari } \\
\newline
Download all Latex-tikz codes from 
\framebox[1.1\width]{ https://github.com/cs19resch11004/hari} 

\section{Problem}
Find the direction vectors and and y-intercepts of the following lines \\
\begin{align} 
    		\myvec{ 1 & 7 } \vec{X} =  0
\end{align}
\begin{align} 
    		\myvec{ 6 & 3 } \vec{X} =  5
\end{align}
\begin{align} 
    		\myvec{ 0 & 1 } \vec{X} =  0
\end{align}
\section{Explanation}
The Slope of a straight line is useful to know the angle made by the given line w.r.t X -axis, simply it indirectly focuses on direction of give line. Similarly Direction Vector also focus on direction of given line. To find out the Direction Vector, so we have to take two different points in the form positional vectors (say $\vec{A}$ and $\vec{B}$) on the line and then find the $\vec{A}$ - $\vec{B}$ to obtain Direction Vector, let say $\vec{D}$. There exist number of Direction Vectors are possible for the same line. If a vector $\vec{D}$ is Direction Vector for the line L1, then $\vec{D}$ also act as Direction Vector for all lines which are parallel to L1.
\\
\\
Let $ ax+by+c=0 $ be a line, then its normal vector and Direction Vectors respectively, let say \begin{align}
	\vec{N} = \myvec{ a \\ b }\\
	\vec{D} = \myvec{ b \\ -a }
\end{align}
\\
Y-intercept of a straight line $ y = mx + c $ is c.
\\
\\
1) $ x  + 7y =  0,$
\\
Normal vector $\vec{N}$ of $ x  + 7y = 0 $ is   \begin{align}
	\vec{N} = \myvec{ 1 \\ 7 }
\end{align}
\\
	Two different points on the given line in the form positional vectors are (say $\vec{A}$ = \myvec{ 7 \\ -1 } and $\vec{B}$ = \myvec{ 0 \\ 0 }).\\
Direction Vector $\vec{D}$ = $\vec{A}$ - $\vec{B}$ = \myvec{ 7 \\ -1 } - \myvec{ 0 \\ 0 } = \myvec{ 7 \\ -1 }
\\
\\
Y-intercept of a straight line $ x  + 7y =  0 $ is 0
\\
\\
\\
2) $ 6x  + 3y =  5,$
\\
Normal vector $\vec{N}$ of $ 6x  + 3y = 5 $ is   \begin{align}
	\vec{N} = \myvec{ 6 \\ 3 }
\end{align}
\\
	Two different points on the given line in the form positional vectors are (say $\vec{A}$ = \myvec{ 0 \\ 5/3 } and $\vec{B}$ = \myvec{ 5/6 \\ 0 }).\\
Direction Vector $\vec{D}$ = $\vec{A}$ - $\vec{B}$ = \myvec{ 0 \\ 5/3 } - \myvec{ 5/6 \\ 0 }) = \myvec{ -5/6 \\ 5/3 }
\\
\\
Y-intercept of a straight line $ 6x  + 3y =  5 $ is 5/3
\\
\\
\\
3) $ y =  0,$
\\
Normal vector $\vec{N}$ of $ y = 0 $ is   \myvec{ 0 \\ 1 }
\\
\\
	Two different points on the given line in the form positional vectors are (say $\vec{A}$ = \myvec{ 5 \\ 0 } and $\vec{B}$ = \myvec{ 2 \\ 0 }).\\
Direction Vector $\vec{D}$ = $\vec{A}$ - $\vec{B}$ = \myvec{ 5 \\ 0 } - \myvec{ 2 \\ 0 } = \myvec{ 3 \\ 0 }
\\
\\
Y-intercept of a straight line $ y =  0 $ is 0
\end{document}
